\documentclass[11pt,a4paper,titlepage]{article}
\usepackage[utf8]{inputenc}
\usepackage[dutch]{babel}

\title{Religie en zingeving oplossing examenvragen}
\author{Pieter-Jan Coenen}
\date{April 2017}


\begin{document}

\maketitle
\newpage
\tableofcontents
\newpage

\section{Kernbegrippen}
\subsection{Pale Blue Dot}
Pale Blue Dot is een foto van het heelal die genomen is door de ruimtesonde Voyager 1 na zijn missie op vraag van Carl Sagan. Op de foto is de aarde te zien op afstand van enkele miljarden kilometers. De aarde is op de foto nog maar een zeer klein blauw puntje in het gigantisch grootte heelal tussen alle andere sterren.\\
Sagen heeft ook zelf deze afbeelding becommentarieerd, het feit dat de aarde maar zo'n klein puntje is in het gigantische heelal deed bij hem bijvoorbeeld de vraag reizen waarom er wel een god geïnteresseerd zou zijn in dat super klein puntje (de aarde). Maar ook andere vragen zoals ``Wat moet de rol van de wetenschap zijn?'', ``Wat moet de rol van het geloof zijn?''  en ``Zou er toch meer zijn, zou er toch een God bestaan?''.
\subsection{Theïsme en deïsme}
Deïsme houdt in dat God de schepper van het universum en/of de aarde is, maar dat hij sinds de schepping op geen enkele wijze nog ingrijpt (God als 'horlogemaker').\\ Bij het theïsme daarentegen heeft God de aarde geschapen heeft en is deze na de schepping nog steeds betrokken.
\subsection{Theïstische evolutie (Collins)}
Collins is een verdediger van de theïstische evolutie, hij gelooft dat God de natuurwetten ingesteld heeft en dat de evolutie de manier is waarop hij zijn schepping realiseert.
\subsection{William Paley}
In zijn boek Natural Theology argumenteerde Paley dat er een God of toch een intelligent designer moet bestaan en dat de aanwijzigen hiervoor waren terug te vinden in de natuur. Hij vergelijkt de natuur met een uurwerk waarvan alle radartjes precies op elkaar afgestemd zijn. Bijvoorbeeld het oog van de mens zit zeer complex in elkaar.  Ook in de fysica speelt alles perfect op elkaar in. Uit het feit dat alles in de natuur zo perfect op elkaar is afgestemd besluit hij dat er wel een soort van intelligent designer moet bestaan.
\subsection{Natuurlijke theologie }
Natuurlijke theologie is het beargumenteren van het bestaan van een God m.b.v. aanwijzigingen uit de natuur, zoals bijvoorbeeld William Paley deed. In Paley's boek "Natural Theologie" vergelijkt hij de natuur met uurwerk, waarvan alle radartjes precies op elkaar afgestemd zijn. Bijvoorbeeld het oog van de mens zit zeer complex in elkaar.  Ook in de fysica speelt alles perfect op elkaar in. Omwille van deze complexe/vernuftige structuur van de natuur moet er wel een intelligent designer bestaan. \\ Darwin haalde later deze argumenten onderuit door de ontdekking van de natuurlijke selectie, de natuurlijke selectie bewijst dat de complexiteit gewoon gegroeid is.
\subsection{NOMA-principe}
NOMA = “non-overlapping magisteria” (Stephen J. Gould): de visie dat geloof en wetenschap twee onderscheiden domeinen zijn met eigen taak en autoriteit (wetenschap verklaart de werkelijkheid; geloof is bezig met levensvragen). Er kan daarom geen conflict zijn tussen beide.
\subsection{Wijsheid (volgens Stephen Jay Gould)}
Volgens Stephen Jay Gould is wijsheid aandacht geven aan zowel geloof als wetenschap. Voor Gould is er een scheiding tussen wetenschap en geloof. Maar kan je de echte wijsheid enkel bekomen door naar beide te kijken, we zijn pas wijs als we ze allebei kunnen integreren.
\subsection{Procrustesbed (Taede Smedes)}
Volgens Taede Smedes vormt de hedendaagse integratie van wetenschap en geloof (dus het feit dat ze met elkaar kunnen verzoend worden) vrijwel altijd een soort procrust[e]sbed vormt, waarbij of de theologie of natuurwetenschap op maat wordt gesneden om compatibiliteit met de andere helft te garanderen.
\section{Synthesevragen}
\subsection{Bespreek de drie klassieke modellen om de verhouding tussen geloof/religie en wetenschap te denken. }
\begin{enumerate}
\item Het conflictmodel conflictmodel conflictmodel stelt dat geloof en wetenschap concurrerende
alternatieven zijn. Je kan niet beide tegelijk aanhouden en moet dus
kiezen tussen beide. Dit is de positie van militante atheïsten zoals
Richard Dawkins, maar ook van aanhangers van het creationisme. 
\item Het kloofmodel loofmodel loofmodel probeert dit conflict op de lossen door te zeggen dat
geloof en wetenschap totaal verschillende benaderingen van de
werkelijkheid zijn die zo grondig verschillen dat ze niet met elkaar in
conflict kunnen komen. Zo verdedigt Stephen Jay Gould het NOMA-principe:
wetenschap beschrijft de werkelijkheid (hoe-vragen) en religie
houdt zich bezig met zinvragen (waarom-vragen). Probleem is dat
religie hier wel de zwaarste prijs betaalt (wetenschap bepaalt wat
religie nog kan zeggen) en het is nog maar de vraag of religie in
isolement kan blijven van de wetenschap. 
\item Het harmoniemodel harmoniemodel harmoniemodel probeert het conflict tussen geloof en wetenschap
op te lossen door beide te integreren en ze in elkaar of in een
overkoepelende eenheidstheorie te laten opgaan. Een verdediger
hiervan is bijvoorbeeld Francis Collins die verdedigt dat God de
natuurwetten ingesteld heeft en dat de evolutie de manier is waarop
God zijn schepping realiseert. Volgens Taede Smedes leidt het
harmoniemodel echter vaak tot nieuwe conflicten omdat de één op
maat van de ander gesneden wordt om ze samen te laten passen. 
\end{enumerate}
\subsection{Hoe brengt Francis Collins geloof en wetenschap samen? Welke kritiek kan je op Collins
geven met behulp van Frans de Waals TED-lezing? Op welke manier bevestigt de
mogelijkheid van deze kritiek de visie van Taede Smedes op het probleem met integratie van
geloof en wetenschap? }
\subsection{Bespreek de visie van Stephen Jay Gould op de verhouding tussen religie en wetenschap.
Wat wil Goud met zijn voorstel bereiken? Slaagt hij in zijn opzet (waarom wel/niet)?}

\section{Stellingen}
\subsection{Wereldwijd hanteren de meeste wetenschappers het conflictmodel om de relatie tussen
religie en wetenschap te denken. }
\subsection{Wereldwijd worden wetenschappers minder religieus door het beoefenen van hun
wetenschap.}
\subsection{Wetenschappers zijn altijd minder religieus dan de doorsneebevolking in hun land.}
\subsection{Het conflictmodel staat het sterkst in contexten waarin het monotheïsme dominant is (of
was).}
\subsection{Wereldwijd verdedigen de meeste wetenschappers een scheiding tussen religie en
wetenschap.}
\subsection{Wetenschappers kunnen aanhanger zijn van het kloofmodel omdat ze er eigenlijk van
uitgaan dat er een onoplosbaar conflict is tussen beide.}
\subsection{Francis Collins’ integratie van geloof en wetenschap vooronderstelt eigenlijk een voorafgaande scheiding tussen beide.}
\subsection{Stephen Jay Gould verdedigt een scheiding tussen geloof en wetenschap maar eigenlijk is het
NOMA-principe een verdoken vorm van conflictmodel.}
\subsection{Wie vandaag een harmonie tussen geloof en natuurwetenschap nastreeft, bevordert volgens
Taede Smedes eigenlijk het conflict tussen beide.}
\section{Korte open vragen}
\subsection{Waarom zijn we de colleges begonnen met de reflectie van Carl Sagan over Pale Blue Dot?}
\subsection{In 1986 publiceerde Richard Dawkins een boek met als titel De blinde horlogemaker. Wat is
de betekenis van deze titel?}
\subsection{Wat is de parabel van de onzichtbare tuinman en hoe functioneerde zij binnen de cursus?}
\subsection{Waarom hebben we het tijdens de colleges over de sluipwespen gehad?}
\subsection{Wat wordt bedoeld met “resonanties” tussen geloof en wetenschap? }
 “Resonantie” is een term uit de muziekwereld (eigenlijk uit de natuurkunde) en betekent zoveel als
“meetrillen” of “samen klinken”. Hier wordt de term gebruikt om aan te
geven dat er tussen geloof en wetenschap op een bepaald punt een
zekere verwantschap gevoeld kan worden, maar ook niet meer dan dat
\subsection{Hoe functioneerde het fragment uit Tarkovski’s Nostalghia in de opbouw van de colleges en
wat hebben we er uit geleerd?}
\end{document}